\freesection{Введение}

Сегодня задачи оптимизации получили чрезвычайно широкое распространение в технике, экономике, управлении. Типичными областями применения теории оптимизации являются прогнозирование, планирование промышленного производства, управление материальными ресурсами, контроль качества выпускаемой продукции, проектирование технологических линий (процессов), а также проектирование агрегатов и технических систем.

Успешность решения подавляющего большинства экономических задач зависит от наилучшего, наивыгоднейшего способа использования ресурсов. И от того, как будут распределены эти, как правило, ограниченные ресурсы, будет зависеть конечный результат деятельности.

Задачи оптимизации большой размерности характеризуются высокой трудоемкостью. Нередко для их решения ресурсов традиционных однопроцессорных компьютеров оказывается недостаточно. Поэтому необходимо использовать доступный ресурс параллелизма современных вычислительных систем для ускорения поиска решения.

У начал разработки метода эллипсоидов стояли такие ученые, как Шор~Н.З., Гершович~В.И., Левин~А.Ю., Юдин~Д.Б., Немировский~А.С., Журбенко~Н.Г., Гулинский~О.В., Поляк~Б.Т.

В дальнейшем идеи данного метода продолжали разрабатывать Стецюк~П.И., Годонога~А.Ф., Донец~Г.А. и др.

Данная работа посвящена описанию метода эллипсоидов и его приложению для решения задач оптимизации большой размерности. Приводится описание схемы метода и особенности его применения для практической задачи оптимизации. Рассматриваются проблемы эффективной реализации алгоритма (разработки программного обеспечения), ориентированного на многопроцессорные и многоядерные вычислительные системы с общей разделяемой памятью. Производится сравнение полученной реализации с существующими непараллельными реализациями метода эллипсоидов.

Таким образом, \textbf{объект} исследования в данной работе -- метод эллипсоидов, \textbf{предмет} -- параллельная реализация метода для ускоренного решения задач оптимизации большой размерности.
\textbf{Задачами} работы являются:

\begin{enumerate}
	\item разработка параллельной реализации метода эллипсоидов;
	\item проверка и тестирование разработанного программного обеспечения;
	\item проведение вычислительных экспериментов с полученной программной реализацией;
	\item приложение параллельной реализации метода к задаче оптимизации большой размерности.
\end{enumerate}

В итоге будет разработано соответствующее программное обеспечение, будут проведены и проанализированы вычислительные эксперименты.

В первой главе рассматриваются теоретические основы...

Во второй главе приведены данные, необходимых для...

В третьей главе приводится краткое описание...

В четвертой главе построена...

В пятой главе приводятся результаты вычислительных экспериментов...