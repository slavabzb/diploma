\freesection{Заключение}

В данной работе были приведены основные положения, выведенные из идей физической экономики. На основе этих положений была построена модель экономической системы древнего общества скотоводов-земледельцев.

В дипломной работе выполнены следующие \textbf{задачи}:
\begin{enumerate}
	\item построена имитационная модель экономической системы древнего общества скотоводов-земледельцев;
	\item проведен имитационный эксперимент;
	\item проверены теоретические положения.
\end{enumerate}

Можно сделать следующие выводы после проведения симуляционного эксперимента в среде VisSim:

\begin{enumerate}
	\item модель экономической системы древнего общества скотоводов-зем\-ле\-дель\-цев является жизнеспособной и правдоподобно описывает поведение древней человеческой общины.
	\item модель дает возможность объяснить экономическую сущность исторических фактов относительно древних общин периода неолита;
	\item проверен принцип увеличения доли свободного времени в общем фонде социального времени по ходу развития общины.
\end{enumerate}

Данная модель может быть улучшена путем более точного описания различных хозяйственных процессов, происходивших в экономике общины. Также возможно применение тензорной методологии для описания этих хозяйственных процессов, при этом уравнения примут более понятный внешний вид, не утратив своего содержания.